\newpage
{\let\clearpage\relax \chapter{正文工具}}

\section{目录}

\section{脚注}

脚注是对正文中词语的补充说明。系统提供的脚注命令如下,序号用于自行设定脚注序号,通常不需要给出。

\begin{latex}
\footnote[number]{text}
\end{latex}

例如,为本文作者\footnote{邹思宇,男,\LaTeX 爱好者}添加脚注。

如果要在脚注中输入带反斜杠的字符串,可使用等宽字体命令加字符串命令输入\footnote{\texttt{脚注命令\string\footnote}}。代码如下。如果需要更多的设置,可以调用脚注宏包\emph{footmisc},对脚注命令\verb|\footnote|进行扩展功能。

\begin{latex}
\footnote{\texttt{\string\footnote}}
\end{latex}

\section{边注}

\LaTeX 本身提供边注命令:

\begin{latex}
\marginpar[左边注]{右边注}
\end{latex}

边注测试\marginpar[这是一个边注]{这是边注啊}。

调用\emph{marginnote}宏包,新定义一个边注。使用\verb|\bz|调用,将会在与段落平齐的地方生成一个边注。例如:

\bz{从这一行开始是用于重新定义边注的代码}

\begin{latex}
% 边注和索引,来自重庆大学LaTeX团队
\renewcommand*{\marginfont}{\color{Note}\sffamily\heiti}
\DeclareDocumentCommand{\bz}{s o m}{%
    \IfBooleanTF {#1}
    {%ture
        \IfNoValueTF{#2}{\marginnote[#3]{#3}}{\marginnote[#2]{#3}}
    }{%false
    \IfNoValueTF{#2}{\marginnote[#3]{#3}}{\marginnote[#2]{#3}}
    \index{#3}
}%
}
\end{latex}

\section{参考文献}

中文著作肯定要符合《GB7714-2015信息与文献参考文献著录规则》的要求,我习惯使用biblatex来生成参考文献。在导言区或者自定义的类文件中添加如下1--5行的代码,调用biblatex宏包并指定bib数据库路径\footnote{文中采用的是相对路径,即数据库为我编译的tex文件的同一目录下的Zousiyu.bib文件}和名称。在正文中使用\footnote{一般写在\texttt{\string\end\{document\}}之前}第7行代码打印参考文献。

本书主要参考了刘海洋\cite{刘海洋}和胡伟\cite{胡伟}编写的教程。使用的参考文献样式是胡振震编写的,源码托管在\href{https://github.com/hushidong/biblatex-gb7714-2015}{Github∙hushidong/biblatex-gb7714-2015}上。

\begin{latex}
\usepackage[
    backend=biber,%处理方式
    style=gb7714-2015%样式
    ]{biblatex}
\addbibresource{Zousiyu.bib}

\printbibliography%打印参考文献
\end{latex}

bib参考文献数据格式如下所示,为分字段显示。各字段可以顾名思义,第一行的“刘海洋”是参考文献标识,你在文中引用参考文献时需要使用此标识。

\begin{latex}
@book{刘海洋,
    title={LATEX入门},
    author={刘海洋},
    publisher={电子工业出版社},
    year={2013},
}
\end{latex}

参考文献使用范例,单独列出\cite{刘海洋}\cite{胡伟},一起列出\cite{刘海洋,胡伟}

范例中使用参考文献标识引用参考文献,具体实现如下。

\begin{latex}
单独列出\cite{刘海洋}\cite{胡伟}
一起列出\cite{刘海洋,胡伟}
\end{latex}

\section{链接}
这部分内容主要用\emph{hyperref}宏包来实现。

\section{引用功能}\label{tools-ref}
在论文写作中,章节、插图、表格、公式和文本经常要前后调整或增添删减,这些引用的位置难以一次确定,所以不能进行直接编号。\LaTeX 提供很智能的方法来解决这个问题,你不用担心引用的编号问题,只管引用就好了,\LaTeX 系统会帮你编号。

在你的导言区添加如下代码,重新定义自动引用的名字。
\begin{latex}
\AtBeginDocument{%
    \def\figureautorefname{图}
    \def\tableautorefname{表}
    \def\partautorefname{卷}
    \def\appendixautorefname{附录}
    \def\equationautorefname{式}
    \def\Itemautorefname{列表}
    \def\chapterautorefname{章}
    \def\sectionautorefname{节}
    \def\subsectionautorefname{小节}
    \def\subsubsectionautorefname{条目}
    \def\paragraphautorefname{自然段}
    \def\Hfootnoteautorefname{脚注}
    \def\AMSautorefname{式}
    \def\theoremautorefname{定理}
    \def\pageautorefname{页}
}
\end{latex}

我们可以使用命令引用一个表格、公式、图片等。如使用如下命令分别引用一张表和一个带编号的公式。引用结果:如\autopageref{tools-ref},\autoref{tools-ref}中\autoref{tools-equation},\autoref{tools-tabular},\autoref{tools-figure}所示。

\begin{latex}
\ref{tools-equation}
\ref{tools-tabular}
\end{latex}

\begin{equation}\label{tools-equation}
\int \arccsc x\,dx=x \arccsc x+\ln(x+\sqrt{x^2-1}+C)
\end{equation}

\begin{table}[!ht]
\begin{center}
    \caption{\TeX 家族标识符}
    \label{tools-tabular}
    \begin{tabular}{|C{10mm}|C{10mm}|}
        \hline
        \multicolumn{2}{|c|}{\TeX 家族标识符}\\
        \hline
        \LaTeX & \LaTeXe\\
        \hline
        \TeX & \XeLaTeX\\
        \hline
    \end{tabular}
\end{center}
\end{table}

\begin{figure}[!htb]
    \begin{center}
        \includegraphics[width=8cm]{tools-figure}
        \caption{Demo of bar plot on a polar axis}
        \label{tools-figure}
    \end{center}
\end{figure}

\section{列表}
\LaTeX 自带的列表环境可调整的样式很有限,调整起来也很麻烦。所以最好直接用别人写好的宏包来调整列表环境。记住一句话,要随心所欲定制\LaTeX 输出的样式,就要用自由度最高的宏包,不要嫌麻烦,否则达不到想要的定制效果。enumitem宏包在定制列表环境方面做得很不错,可调样式很多,能满足大部分需求。借用wklchris\footnote{\url{https://github.com/wklchris/Note-by-LaTeX}}绘制的enumitem列表长度参数图。

\begin{figure}[!htb]
    \centering
    \includegraphics[width=10cm]{enumitem-parameters}
    \caption{列表长度参数图}
\end{figure}

enumitem提供的参数很多,想每一项都弄明白得花点时间,我不解释每一项参数,而是从例子开始入手。看第一个例子,让列表标签和正文段落一样缩进两个字符。labelindent控制缩进两个字符,当标签很长的时候,缩进的距离会很奇怪,所以需要设置leftmargin的值为自动调整。

\begin{codeshowabove}
~~四十年来家国,三千里地山河。凤阁龙楼连霄汉,玉树琼枝作烟萝,几曾识干戈?一旦归为臣虏,沈腰潘鬓消磨。最是仓皇辞庙日,教坊犹奏别离歌,垂泪对宫娥。
\begin{enumerate}[label=(\arabic*).,labelindent=\parindent]
    \item 奇怪的缩进
    \item 奇怪的缩进
\end{enumerate}
\begin{enumerate}[label=(\arabic*),labelindent=\parindent,leftmargin=*]
    \item 第一项
    \item 第二项
\end{enumerate}
\end{codeshowabove}

中文文章的要求一般是,序号前缩进两个字符,列表项目之间无额外行距,列表换行后无缩进。对itemsep、topsep赋值,分别消除列表项目之间的间距、列表与上下文(正文)之间的间距;对leftmargin赋值,消除列表项目换行后的缩进,对labelindent赋值,控制序号前缩进两个字符;因为标签长度不太可控,itemindent的值不好计算,所以设置其值为自动计算比较稳妥。

四十年来家国,三千里地山河。凤阁龙楼连霄汉,玉树琼枝作烟萝,几曾识干戈?一旦归为臣虏,沈腰潘鬓消磨。最是仓皇辞庙日,教坊犹奏别离歌,垂泪对宫娥。
\begin{enumerate}[label=(\arabic*),itemsep=0pt,parsep=0pt,topsep=0pt,leftmargin=0pt,labelindent=\parindent,itemindent=*]
\item 四十年来家国,三千里地山河。凤阁龙楼连霄汉,玉树琼枝作烟萝,几曾识干戈?一旦归为臣虏,沈腰潘鬓消磨。最是仓皇辞庙日,教坊犹奏别离歌,垂泪对宫娥。

四十年来家国,三千里地山河。凤阁龙楼连霄汉,玉树琼枝作烟萝,几曾识干戈?一旦归为臣虏,沈腰潘鬓消磨。最是仓皇辞庙日,教坊犹奏别离歌,垂泪对宫娥。

\item 四十年来家国,三千里地山河。凤阁龙楼连霄汉,玉树琼枝作烟萝,几曾识干戈?一旦归为臣虏,沈腰潘鬓消磨。最是仓皇辞庙日,教坊犹奏别离歌,垂泪对宫娥。
\item 四十年来家国,三千里地山河。凤阁龙楼连霄汉,玉树琼枝作烟萝,几曾识干戈?一旦归为臣虏,沈腰潘鬓消磨。最是仓皇辞庙日,教坊犹奏别离歌,垂泪对宫娥。
\end{enumerate}

四十年来家国,三千里地山河。凤阁龙楼连霄汉,玉树琼枝作烟萝,几曾识干戈?一旦归为臣虏,沈腰潘鬓消磨。最是仓皇辞庙日,教坊犹奏别离歌,垂泪对宫娥。

\subsection{常规列表}
常规列表环境。

\begin{codeshow}
\begin{itemize}
    \item[记号] 条目1
    \item[-] description1
    \item[*] description2
\end{itemize}
\end{codeshow}

常规列表的条目之间间距较大,可以使用长度赋值命令将条目环境额外的垂直空白设置为0pt,达到与正文间距一致。

\begin{latex}
\itemsep=0pt
\parskip=0pt
\end{latex}

\subsection{排序列表}
排序列表的基本形式。

\begin{codeshow}
    \begin{enumerate}
        \item 条目1
        \item 条目2
        \item 条目3
    \end{enumerate}
\end{codeshow}

排序列表可以嵌套,各层条目序号不一,我们可对其序号、标号和前缀进行重新定义,以生成所需要的条目样式。但是重新定义命令使用起来麻烦,排序列表默认命令也复杂,不便记忆,更不便于重定义。

为了方便,我们直接使用\emph{paralist}宏包,我们只需要将标号样式填入方括号中,即可对标号进行重定义。此法在后面说明。

\subsection{解说列表}
示例如下,该类型列表用于对专业术语进行解释。具体设置不做详细说明,因为使用不便,后面有更好的宏包可以对以上所提到的三类列表进行更简便地进行设置。

\begin{codeshow}
    \begin{description}
        \item[APLL] 
        Automatic Phase-Locked Loop
        自动锁相环
        \item[GPS] Global 
        Positioning System 全球定位系统
        \item[SPACETRACK] 
        Space Tracking 空间跟踪
    \end{description}
\end{codeshow}

\subsection{带圈数字列表}

在许多文章中,特别是中文文章中,我们会见到带有圆圈的数字。它们有点是单独出现的,有点作为列表的计数出现。这里给出一个利用TikZ绘制的方法,\textbf{既能在正文中调用,也能在列表中调用}。基本的思路是定义一个新命令,接受一个数字参数,用 TikZ 在它周围画圈。同时要考虑基线和对齐的问题。代码实现如下\footnote{此法来源于\href{http://tex.stackexchange.com/questions/7032/good-way-to-make-textcircled-numbers}{tikz pgf - Good way to make \cmd{textcircled} numbers? - TeX - LaTeX Stack Exchange}}:

\begin{latex}
\usepackage{tikz}
\usepackage{etoolbox}
\newcommand{\circled}[2][]{\tikz[baseline=(char.base)]
    {\node[shape = circle, draw, inner sep = 1pt]
        (char) {\phantom{\ifblank{#1}{#2}{#1}}};%
        \node at (char.center) {\makebox[0pt][c]{#2}};}}
\robustify{\circled}
\end{latex}

这个新定义的命令可以按照\cmd{textcircled}方法在正文中使用。

\begin{codeshow}
Numbers aligned with the text:  \circled{1} \circled{2} \circled{3} end.
\end{codeshow}

如果需要用在列表中,则因为「脆弱命令」的问题,需要处理一下。这里我们选择使用etoolbox宏包提供的\cmd{robustify}来处理一下,同时结合 enumitem 宏包,给出示例用法如下:

\begin{codeshow}
\begin{enumerate}[label=\dcircled{\arabic*}, noitemsep]
    \item 力微任重久神疲,再竭衰庸定不支
    \item 苟利国家生死以,岂因祸福避趋之
    \item 谪居正是君恩厚,养拙刚于戍卒宜
    \item 戏与山妻谈故事,试吟断送老头皮
\end{enumerate}
\end{codeshow}


\section{附录}

\section{代码环境}

首先载入\emph{listings}宏包,定义基础代码环境,我取名为\emph{CodeBase},这个基础代码环境定义的样式能被后续的代码环境调用,免去重复设置。也正是因为基础代码环境的通用性,所以这里只适合定义在所有代码环境中都适用的样式,如字体、各种边距、换行和标识等。

\begin{latex}
\lstdefinestyle{CodeBase}
{
    basicstyle=\small\ttfamily,
    frame=l,
    aboveskip=0pt,%上边距
    belowskip=0pt,%下边距
    lineskip=0pt,
    tabsize=4,%设置tab空格数
    showtabs=false,%Tab
    showspaces=false,%空格标识
    showstringspaces=false,
    numbers=left,
    numbersep=5pt,%行号与代码距离
    numberstyle=\small\ttfamily,
    rulecolor=\color{cyan},
    boxpos=c,
    xleftmargin=1em,%左边距
    xrightmargin=0pt,
    breaklines=true,%自动换行
    breakindent=0pt,%换行后缩进为0
    extendedchars=false,%解决代码跨页时,章节标题,页眉等汉字不显示的问题
    framesep=3pt,
    rulesep=2pt,
    framerule=1pt,
    %代码颜色设置
    backgroundcolor=\color{gray!5},
    stringstyle=\color{green!40!black!100},
    keywordstyle=\bfseries\color[RGB]{0,0,255},
    commentstyle=\slshape\color{black!60},
}
\end{latex}

接下来,我们就可以用这个基本样式来定义一个专用于\LaTeX 代码书写的样式和相应的环境。

\begin{latex}
%LaTeX代码环境用
\lstdefinestyle{LaTeX}
{
    style=CodeBase,
    language=[LaTeX]TeX,
    classoffset=0,
    morekeywords={addplot, begin, end},
}

%定义latex代码专用环境
\lstnewenvironment{latex}[1]{\lstset{style=LaTeX}}{}
\end{latex}

最后,直接在正文中使用新定义的环境\emph{latex}框住所需要展示的代码即可。

上面定义了一个\LaTeX 专用的代码环境,实际使用肯定不只\LaTeX 代码需要展示,还有诸如\emph{Python},\emph{MATLAB}等大量其他代码需要展示。这里我们在定义一个用于展示\emph{MATLAB}代码的环境,同样也是从基础样式\emph{CodeBase}进行衍生,只需要几条简单的命令即可。

\begin{latex}
%matlab代码展示
\lstdefinestyle{Matlab}{
    style=CodeBase,
    language=Matlab
}

%定义Matlab代码专用环境
\lstnewenvironment{Matlab}[1]{\lstset{style=Matlab}}{}
\end{latex}

MATLAB代码高亮测试。

\begin{Matlab}{}
t=0:pi/10:2*pi;
[X,Y,Z]=cylinder(4*cos(t));
subplot(1,2,1);mesh(X);title('X');
subplot(1,2,2);mesh(Y);title('Y');
\end{Matlab}

从\emph{CodeBase}定义的新样式X,其设置可以覆盖\emph{CodeBase}中的设置,如下面这段\emph{Python}代码高亮测试中,我们在代码中定义了一句\verb|keywordstyle=\slshape\color[RGB]{0,0,255},|,让\emph{Python}代码中的关键词变为斜体,其他代码环境不受影响。

\begin{latex}
\lstdefinestyle{python}{
    style=CodeBase,
    keywordstyle=\slshape\color[RGB]{0,0,255},%%%%%%%就是这句
    language=Python,
    morekeywords={def},
}
\lstnewenvironment{python}[1]{\lstset{style=python}}{}
\end{latex}

\emph{Python}代码展示。

\begin{python}{}
def ffmpeg_concat_av(files, output, ext):
    print('Merging video parts... ', end="", flush=True)
    params = [FFMPEG] + LOGLEVEL
    for file in files:
        if os.path.isfile(file): params.extend(['-i', file])
    params.extend(['-c:v', 'copy'])
    if ext == 'mp4':
        params.extend(['-c:a', 'aac'])
    elif ext == 'webm':
        params.extend(['-c:a', 'vorbis'])
    params.extend(['-strict', 'experimental'])
    params.append(output)
    return subprocess.call(params)
\end{python}

\emph{listings}宏包识别的代码关键词肯定是有限的,但好在它提供一个参数可以扩充关键词。比如我们为\emph{c++}语言添加更多的关键词,只需要在设置里面写下如下代码。关键词想要多少都行,依据实际情况补充。

\begin{latex}
\lstset{
    morekeywords={alignas,continute,friend,register,true,alignof,decltype,goto,reinterpret_cast,try,asm,defult,if,return,typedef,auto,delete,inline,short,typeid,bool,do,int,signed,typename,break,double,long,sizeof,union,case,dynamic_cast,mutable,static,unsigned,catch,else,namespace,static_assert,using,char,enum,new,static_cast,virtual,char16_t,char32_t,explict,noexcept,struct,void,export,nullptr,switch,volatile,class,extern,operator,template,wchar_t,const,false,private,this,while,constexpr,float,protected,thread_local,const_cast,for,public,throw,std}
},
\end{latex}


