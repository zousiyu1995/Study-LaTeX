\newpage
\chapter{版面和格式}
\thispagestyle{chapterpage}

\section{文本格式}

\LaTeX 将多个空格视为一个,多个换行也会被视为一个。一般习惯使用\lstinline|~|产生一个空格,使用\lstinline|mbox{}|产生一个空白段落(实际上就是一个空白行),使用\lstinline|\par|产生一个带缩进的新段,使用\lstinline|\\|来强制换行,但下一段的缩进会消失。

段落之间的距离一般这样控制:

\begin{latex}
\setlength{\parskip}{0pt plus 1pt}%默认值
\end{latex}

用\lstinline|\newpage|命令开始新的一页。

用\lstinline|\clearpage|命令清空浮动体队列5,并开始新的一页。

用\lstinline|\cleardoublepage|命令清空浮动体队列,并在偶数页上开始新的一页。
注意:以上命令都是基于\lstinline|\vfill|的。如果要连续新开两页,请在中间加上一个空的箱,如:

\begin{latex}
\newpage\mbox{}\newpage
\end{latex}

\LaTeX 默认使用两端对齐来排版,我们可以用\lstinline|\flushleft,\flushright,\center|这三个环境来构造居左,居右,居中三种版式。特殊情况可以使用\lstinline|\centering,\raggedleft,\raggedright|来实现居中,居右,居左。

\section{标题}

\section{页眉页脚}
\subsection{版式}
\LaTeX{}系统包含如下几种常用的页面样式:
\label{plain}
\begin{description}
    \item[empty] 无页眉页脚。
    \item[plain] 无页眉,页脚有居中的页码,report和article文类默认版式。
    \item[headings] 偶数页页眉左端是页码,右端是章标题;奇数页页眉左端是节标题,右端是页码;章标题页的板式为plain,book文类默认版式。
    \item[myheadings] 自定义的页眉页脚样式,\textbf{不推荐,我们还是用宏包比较方便}。
\end{description}

一般使用系统命令\lstinline|\pagestyle{code}|设置板式。写在导言区会设置全局版式,写在正文区会设置当前页及后续页面的版式,\textbf{推荐写在导言区}。在正文中设置当前页版式一般用\lstinline|\thispagestyle{code}|,该命令不影响后续页面版式。例如,通常使用\lstinline|\thispagestyle{empty}|清空封面页眉页脚。

系统默认的页码计数形式有\lstinline|alph,Alph,arabic,roman,Roman|,使用\lstinline|\pagenumbering{numstyle}|设置页码样式,都能顾名思义。一般而言,论文封面之后——摘要、目录等页码用大写罗马数字,正文及参考文献使用阿拉伯数字。如果需要在某一页重置页码计数器,可在该页源文件处使用命令\lstinline|\setcounter{page}{num}|重置计数器,page是系统定义的页码计数器,num按意愿填写。

另外,book文类提供三条分区命令,\textbf{分区命令之后的区域}分别为序、文、跋。

\begin{latex}
\frontmatter
% 序
\mainmatter
% 文
\backmatter
% 跋
\end{latex}

\subsection{fancyhdr宏包}
一般来说,设置页眉页脚比较广泛的宏包是\pkg{fancyhdr}。该宏包定义了一个额外的版式——fancy。对单页文档来说,其将页眉页脚分为了左中右三个部分,分布用\lstinline|\[lcr]head|和\lstinline|\[lcr]foot|控制。修改版式之前一般习惯性清空原有版式,\textbf{清空版式只需要将命令的参数空置即可}。

\begin{latex}
\usepackage{fancyhdr}
\pagestyle{fancy}
\fancyhf{}%清空页眉页脚
    \lhead{}
    \chead{}
    \rhead{\bfseries zousiyu}
    \lfoot{Leftfoot}
    \cfoot{\thepage}
    \rfoot{Rightfoot}
\end{latex}

对双页文档来说,使用奇数页、偶数页、页眉、页脚、左、中、右组合将页面分为12个区域进行细致的控制,一般采用\lstinline|\fancy[head,foot,hf]|三条命令进行设置,例如:

\begin{latex}
\fancyhead[LE,RO]{\slshape\rightmark}
\fancyhead[LO,RE]{\slshape\leftmark}
\fancyfoot[C]{\thepage}
\end{latex}

\pkg{fancyhdr}宏包常用命令如\autoref{fancyhdr-tabular}~所示:

\begin{table}
    \caption{fancyhdr宏包常用命令}
    \label{fancyhdr-tabular}
\begin{tabular}{ll}
    \toprule
    命令 & 作用\\
    \midrule
    \lstinline|\fancyhead[position]{header}| & 设置页眉版式\\
    \lstinline|\fancyfoot[position]{footer}| & 设置页脚版式\\
    \lstinline|\fancyhf[position]{header/footer}| & 可同时设置页眉页脚,使用H、F指定页眉还是页脚\\
    E & 偶数页,左页\\
    O & 奇数页,右页\\
    L、C、R & 顾名思义,左、中、右\\
    H、F & 页眉、页脚\\
    \lstinline|\rightmark| & 较低级别的标题,section\\
    \lstinline|\leftmark| & 较高级别的信息,chapter\\
    \bottomrule
\end{tabular}
\end{table}

我们可以将章节标题和序号插入到页眉或者页脚中去,其格式与正文中章节标题的定义一样。如果需要更改,要重新定义。例如,可以使用如下代码重新定义页眉内的章标题样式,用在在本书中,这将会使页眉的“第X章 版式”更改为“X 版式”。

具体更改页眉页脚区域章节显示样式的代码如下。

\begin{latex}
\renewcommand{\chaptermark}[1]{\markleft{\thesection.\#1}}
%两种一样,\markleft影响\leftmark,而\makeboth影响两着,需要选一
\renewcommand{\chaptermark}[1]{\markboth{\thechapter.\ #1}{节样式空置表示修改章样式}}
\renewcommand{\chaptermark}[1]{\markboth{章样式}{节样式}}
\end{latex}

在book文件类别下,\lstinline|\leftmark|自动存录各章之章名,\lstinline|\rightmark|记录节标题。所以,想要在页眉上显示章节标题是很容易实现的。

\begin{latex}
\lhead{\leftmark}%左页眉显示章
\rhead{\rightmark}%右页眉显示节
\end{latex}

另外,还可以使用\lstinline|\fancypagestyle{code}{style}|更改系统自带的4种版式或者自定义自己的版式。例如,我们专门给\textbf{章标题}所在页定义一个版式\footnote{章标题所在页默认版式为plain,见\autopageref{plain}},然后在各章标题页手动指定该页版式为\lstinline|chapterpage|,使之与其他页面版式相似。效果见各章章标题所在页。

\begin{latex}
% 章标题所在页专用版式
\fancypagestyle{chapterpage}{%
    \chead{\color{title}\slshape\leftmark}
    \lhead{\color{title}\bfseries\itshape\thepage/\pageref{LastPage}}
    \rhead{\color{title}\sffamily\LaTeXe{}学习笔记}
}
\end{latex}

\section{颜色}
一般来说,我们调用下\emph{xcolor}这个宏包。如果对内置的颜色了解,或者现有\emph{RGB}颜色值,一般使用如下代码直接调用颜色。

\begin{center}
\color[RGB]{204, 128, 92}{Color Text中文测试}
\end{center}

\begin{latex}
\color[RGB]{204, 128, 92}{Color Text中文测试}
\end{latex}

但是每次调用颜色都写颜色代码似乎不方便,我们可以先定义,基本定义形式如下。
\begin{latex}
\usepackage{xcolor}%颜色宏包
\definecolor{backcolor}{RGB}{242,242,242}%背景色
\definecolor{comment}{RGB}{0,128,0}%注释
\definecolor{keyword}{RGB}{0,0,255}%关键词
\definecolor{name名字随意}{model色值类型}{color-spec色值范围}
\end{latex}

然后,我们就可以直接调用我们定义的颜色名称来设定颜色了。

\begin{center}
\color{blue}{\slshape\zihao{3} function, return, if, true, false}
\end{center}

\begin{latex}
\color{keyword}{\slshape function, return, if, true, false}
\end{latex}

\section{标题}

\CTeX 宏包提供标题修改功能,所以中文文档很容易实现标题的修改。如果是直接使用\CTeX 提供的文类,标题是可以直接修改的;如果仅仅只是调用了\CTeX 宏包,需要给宏包加上调用参数才能修改标题。

\begin{latex}
\usepackage[
    heading=true,%启用修改章节标题的接口
]{ctex}
\end{latex}

实际使用中发现\CTeX{}宏包和某个宏包冲突,修改标题间距并不起作用,暂时改用titlesec宏包定制标题样式。

\begin{latex}
\titleformat{\chapter}[hang]%
	{\centering\color{title}\heiti\zihao{1}}%格式
	{第\, \thechapter 节}%标签
	{20pt}%标签和标题间距
	{}%
	[\vspace{5mm}]%
\end{latex}


